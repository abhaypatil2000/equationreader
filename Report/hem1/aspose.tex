\documentclass{article}
\usepackage{amsmath}
\usepackage{color,pxfonts,fix-cm}
\usepackage{latexsym}
\usepackage[mathletters]{ucs}
\DeclareUnicodeCharacter{32}{$\ $}
\DeclareUnicodeCharacter{46}{\textperiodcentered}
\usepackage[T1]{fontenc}
\usepackage[utf8x]{inputenc}
\usepackage{pict2e}
\usepackage{wasysym}
\usepackage[english]{babel}
\usepackage{tikz}
\pagestyle{empty}
\usepackage[margin=0in,paperwidth=604pt,paperheight=806pt]{geometry}
\begin{document}
\definecolor{color_35424}{rgb}{0,0.682353,0.937255}
\definecolor{color_196682}{rgb}{0.658824,0.666667,0.678431}
\definecolor{color_283006}{rgb}{1,1,1}
\definecolor{color_29791}{rgb}{0,0,0}
\definecolor{color_63426}{rgb}{0.133333,0.117647,0.121569}
\definecolor{color_248286}{rgb}{0.862745,0.866667,0.878431}
\begin{tikzpicture}[overlay]
\path(0pt,0pt);
\filldraw[color_35424][even odd rule]
(539.6376pt, -783.5875pt) -- (577.4357pt, -783.5875pt)
 -- (577.4357pt, -783.5875pt)
 -- (577.4357pt, -1.499451pt)
 -- (577.4357pt, -1.499451pt)
 -- (539.6376pt, -1.499451pt) -- cycle
;
\filldraw[color_196682][even odd rule]
(539.6376pt, -783.5894pt) -- (577.4357pt, -783.5894pt)
 -- (577.4357pt, -783.5894pt)
 -- (577.4357pt, -392.6966pt)
 -- (577.4357pt, -392.6966pt)
 -- (539.6376pt, -392.6966pt) -- cycle
;
\filldraw[color_283006][even odd rule]
(521.0304pt, -176.5216pt) -- (579.9701pt, -176.5216pt)
 -- (579.9701pt, -176.5216pt)
 -- (579.9701pt, -77.3313pt)
 -- (579.9701pt, -77.3313pt)
 -- (521.0304pt, -77.3313pt) -- cycle
;
\draw[color_283006,line width=0.96pt,miter limit=1000]
(521.0304pt, -176.5216pt) -- (579.9701pt, -176.5216pt)
 -- (579.9701pt, -176.5216pt)
 -- (579.9701pt, -77.3313pt)
 -- (579.9701pt, -77.3313pt)
 -- (521.0304pt, -77.3313pt) -- cycle
;
\draw[color_29791,line width=0.48pt]
(572.04pt, 10.15997pt) -- (572.04pt, 0.319946pt)
;
\draw[color_29791,line width=0.48pt]
(579.48pt, -7.359985pt) -- (589.32pt, -7.359985pt)
;
\draw[color_29791,line width=0.48pt]
(571.8pt, -786.4pt) -- (571.8pt, -796.24pt)
;
\draw[color_29791,line width=0.48pt]
(580.2pt, -778.24pt) -- (590.04pt, -778.24pt)
;
\end{tikzpicture}
\begin{picture}(-5,0)(2.5,0)
\put(420.6,-67.11902){\fontsize{11}{1}\usefont{T1}{ptm}{m}{n}\selectfont\color{color_63426}R}
\put(427.08,-67.11902){\fontsize{7.7}{1}\usefont{T1}{ptm}{m}{n}\selectfont\color{color_63426}ATIONAL}
\put(456.84,-67.11902){\fontsize{11}{1}\usefont{T1}{ptm}{m}{n}\selectfont\color{color_63426} N}
\put(468.12,-67.11902){\fontsize{7.7}{1}\usefont{T1}{ptm}{m}{n}\selectfont\color{color_63426}UMBERS}
\put(495.24,-67.11902){\fontsize{11}{1}\usefont{T1}{ptm}{m}{n}\selectfont\color{color_63426} }
\end{picture}
\begin{tikzpicture}[overlay]
\path(0pt,0pt);
\filldraw[color_35424][even odd rule]
(499.3992pt, -67.14886pt) -- (507.2995pt, -67.14886pt)
 -- (507.2995pt, -67.14886pt)
 -- (507.2995pt, -59.24945pt)
 -- (507.2995pt, -59.24945pt)
 -- (499.3992pt, -59.24945pt) -- cycle
;
\end{tikzpicture}
\begin{picture}(-5,0)(2.5,0)
\put(507.24,-67.11902){\fontsize{11}{1}\usefont{T1}{ptm}{m}{n}\selectfont\color{color_63426} }
\put(511.32,-67.11902){\fontsize{11}{1}\usefont{T1}{ptm}{m}{n}\selectfont\color{color_63426}1}
\end{picture}
\begin{tikzpicture}[overlay]
\path(0pt,0pt);
\draw[color_29791,line width=0.96pt]
(-15pt, -762.64pt) -- (-15pt, -785.44pt)
;
\draw[color_29791,line width=0.96pt]
(-15pt, -762.64pt) -- (-15pt, -785.44pt)
;
\begin{scope}
\clip
(381.6pt, -718.001pt) -- (381.6pt, -586.5006pt)
 -- (381.6pt, -586.5006pt)
 -- (523.8pt, -586.5006pt)
 -- (523.8pt, -586.5006pt)
 -- (523.8pt, -718.001pt) -- cycle
;
\end{scope}
\end{tikzpicture}
\begin{picture}(-5,0)(2.5,0)
\put(381.6,-718){\includegraphics[width=142.2pt,height=131.5001pt]{latexImage_23dd5bdf486726247459a3cc2feb5491.png}}
\put(51,-280.24){\fontsize{14}{1}\usefont{T1}{ptm}{m}{n}\selectfont\color{color_35424}1.1  Introduction}
\put(51,-303.28){\fontsize{12}{1}\usefont{T1}{ptm}{m}{n}\selectfont\color{color_63426}In Mathematics, we frequently come across simple equations to be solved. For example,}
\put(51,-319.6){\fontsize{12}{1}\usefont{T1}{ptm}{m}{n}\selectfont\color{color_63426}the equation}
\put(214.44,-319.6){\fontsize{12}{1}\usefont{T1}{ptm}{m}{n}\selectfont\color{color_63426}x}
\put(219.96,-319.6){\fontsize{12}{1}\usefont{T1}{ptm}{m}{n}\selectfont\color{color_63426} + 2 =}
\put(251.88,-319.6){\fontsize{12}{1}\usefont{T1}{ptm}{m}{n}\selectfont\color{color_63426}13}
\put(427.56,-319.6){\fontsize{12}{1}\usefont{T1}{ptm}{m}{n}\selectfont\color{color_63426}(1)}
\put(51,-341.68){\fontsize{12}{1}\usefont{T1}{ptm}{m}{n}\selectfont\color{color_63426}is solved when }
\put(120.12,-341.68){\fontsize{12}{1}\usefont{T1}{ptm}{m}{n}\selectfont\color{color_63426}x}
\put(125.4,-341.68){\fontsize{12}{1}\usefont{T1}{ptm}{m}{n}\selectfont\color{color_63426} = 1}
\put(141.24,-341.68){\fontsize{12}{1}\usefont{T1}{ptm}{m}{n}\selectfont\color{color_63426}1, because this value of }
\put(248.04,-341.68){\fontsize{12}{1}\usefont{T1}{ptm}{m}{n}\selectfont\color{color_63426}x}
\put(253.32,-341.68){\fontsize{12}{1}\usefont{T1}{ptm}{m}{n}\selectfont\color{color_63426} satisfies the given equation. }
\put(382.68,-341.68){\fontsize{12}{1}\usefont{T1}{ptm}{m}{n}\selectfont\color{color_63426}The  solution}
\put(51,-358){\fontsize{12}{1}\usefont{T1}{ptm}{m}{n}\selectfont\color{color_63426}1}
\put(56.28,-358){\fontsize{12}{1}\usefont{T1}{ptm}{m}{n}\selectfont\color{color_63426}1 is a }
\put(81.24,-358){\fontsize{12}{1}\usefont{T1}{ptm}{m}{n}\selectfont\color{color_63426}natural number}
\put(159.72,-358){\fontsize{12}{1}\usefont{T1}{ptm}{m}{n}\selectfont\color{color_63426}. On the other hand, for the equation}
\put(214.44,-380.08){\fontsize{12}{1}\usefont{T1}{ptm}{m}{n}\selectfont\color{color_63426}x}
\put(219.96,-380.08){\fontsize{12}{1}\usefont{T1}{ptm}{m}{n}\selectfont\color{color_63426} + 5 =}
\put(251.88,-380.08){\fontsize{12}{1}\usefont{T1}{ptm}{m}{n}\selectfont\color{color_63426}5}
\put(427.56,-380.08){\fontsize{12}{1}\usefont{T1}{ptm}{m}{n}\selectfont\color{color_63426}(2)}
\put(51,-402.4){\fontsize{12}{1}\usefont{T1}{ptm}{m}{n}\selectfont\color{color_63426}the solution gives the }
\put(153,-402.4){\fontsize{12}{1}\usefont{T1}{ptm}{m}{n}\selectfont\color{color_63426}whole number }
\put(227.88,-402.4){\fontsize{12}{1}\usefont{T1}{ptm}{m}{n}\selectfont\color{color_63426}0 (zero). If we consider only natural numbers,}
\put(51,-418.48){\fontsize{12}{1}\usefont{T1}{ptm}{m}{n}\selectfont\color{color_63426}equation (2) cannot be solved. }
\put(188.52,-418.48){\fontsize{12}{1}\usefont{T1}{ptm}{m}{n}\selectfont\color{color_63426}T}
\put(194.76,-418.48){\fontsize{12}{1}\usefont{T1}{ptm}{m}{n}\selectfont\color{color_63426}o solve equations like (2), we added the number zero to}
\put(51.00002,-434.8){\fontsize{12}{1}\usefont{T1}{ptm}{m}{n}\selectfont\color{color_63426}the collection of natural numbers and obtained the whole numbers. Even whole numbers}
\put(51.00002,-450.88){\fontsize{12}{1}\usefont{T1}{ptm}{m}{n}\selectfont\color{color_63426}will not be sufficient to solve equations of type}
\put(208.44,-473.2){\fontsize{12}{1}\usefont{T1}{ptm}{m}{n}\selectfont\color{color_63426}x}
\put(213.96,-473.2){\fontsize{12}{1}\usefont{T1}{ptm}{m}{n}\selectfont\color{color_63426} + 18 =}
\put(251.88,-473.2){\fontsize{12}{1}\usefont{T1}{ptm}{m}{n}\selectfont\color{color_63426}5}
\put(427.56,-473.2){\fontsize{12}{1}\usefont{T1}{ptm}{m}{n}\selectfont\color{color_63426}(3)}
\put(69,-495.28){\fontsize{12}{1}\usefont{T1}{ptm}{m}{n}\selectfont\color{color_63426}Do you see ‘why’? }
\put(157.8,-495.28){\fontsize{12}{1}\usefont{T1}{ptm}{m}{n}\selectfont\color{color_63426}W}
\put(167.88,-495.28){\fontsize{12}{1}\usefont{T1}{ptm}{m}{n}\selectfont\color{color_63426}e require the number –13 which is not a whole number}
\put(416.28,-495.28){\fontsize{12}{1}\usefont{T1}{ptm}{m}{n}\selectfont\color{color_63426}. }
\put(421.08,-495.28){\fontsize{12}{1}\usefont{T1}{ptm}{m}{n}\selectfont\color{color_63426}This}
\put(51,-511.6){\fontsize{12}{1}\usefont{T1}{ptm}{m}{n}\selectfont\color{color_63426}led us to think of }
\put(133.324,-511.6){\fontsize{12}{1}\usefont{T1}{ptm}{m}{n}\selectfont\color{color_63426}integers, (positive and negative)}
\put(294.12,-511.6){\fontsize{12}{1}\usefont{T1}{ptm}{m}{n}\selectfont\color{color_63426}. Note that the positive integers}
\put(51,-527.6799){\fontsize{12}{1}\usefont{T1}{ptm}{m}{n}\selectfont\color{color_63426}correspond to natural numbers. One may think that we have enough numbers to solve all}
\put(51,-544){\fontsize{12}{1}\usefont{T1}{ptm}{m}{n}\selectfont\color{color_63426}simple equations with the available list of integers. Now consider the equations}
\put(224.28,-563.2){\fontsize{12}{1}\usefont{T1}{ptm}{m}{n}\selectfont\color{color_63426}2}
\put(230.28,-563.2){\fontsize{12}{1}\usefont{T1}{ptm}{m}{n}\selectfont\color{color_63426}x}
\put(235.56,-563.2){\fontsize{12}{1}\usefont{T1}{ptm}{m}{n}\selectfont\color{color_63426}  =}
\put(251.88,-563.2){\fontsize{12}{1}\usefont{T1}{ptm}{m}{n}\selectfont\color{color_63426}3}
\put(427.56,-563.2){\fontsize{12}{1}\usefont{T1}{ptm}{m}{n}\selectfont\color{color_63426}(4)}
\put(208.44,-582.4){\fontsize{12}{1}\usefont{T1}{ptm}{m}{n}\selectfont\color{color_63426}5}
\put(214.4401,-582.3997){\fontsize{12}{1}\usefont{T1}{ptm}{m}{n}\selectfont\color{color_63426}x}
\put(219.9601,-582.3997){\fontsize{12}{1}\usefont{T1}{ptm}{m}{n}\selectfont\color{color_63426} + 7 =}
\put(251.8801,-582.3997){\fontsize{12}{1}\usefont{T1}{ptm}{m}{n}\selectfont\color{color_63426}0}
\put(427.5601,-582.3997){\fontsize{12}{1}\usefont{T1}{ptm}{m}{n}\selectfont\color{color_63426}(5)}
\put(51.00009,-604.4796){\fontsize{12}{1}\usefont{T1}{ptm}{m}{n}\selectfont\color{color_63426}for which we cannot find a solution from the integers}
\put(266.5201,-604.4796){\fontsize{12}{1}\usefont{T1}{ptm}{m}{n}\selectfont\color{color_63426}.}
\put(269.1601,-604.4796){\fontsize{12}{1}\usefont{T1}{ptm}{m}{n}\selectfont\color{color_63426} (Check this)}
\put(51.00009,-635.9196){\fontsize{12}{1}\usefont{T1}{ptm}{m}{n}\selectfont\color{color_63426}W}
\put(61.08009,-635.9196){\fontsize{12}{1}\usefont{T1}{ptm}{m}{n}\selectfont\color{color_63426}e need the numbers }
\end{picture}
\begin{tikzpicture}[overlay]
\path(0pt,0pt);
\begin{scope}
\clip
(146.28pt, -615.28pt) -- (158.28pt, -615.28pt)
 -- (158.28pt, -615.28pt)
 -- (158.28pt, -646pt)
 -- (158.28pt, -646pt)
 -- (146.28pt, -646pt) -- cycle
;
\draw[color_29791,line width=0.58824pt,line cap=round,line join=round]
(148.2pt, -630.88pt) -- (155.4pt, -630.88pt)
;
\end{scope}
\begin{scope}
\clip
(148.92pt, -615.52pt) -- (158.28pt, -615.52pt)
 -- (158.28pt, -615.52pt)
 -- (158.28pt, -629.2pt)
 -- (158.28pt, -629.2pt)
 -- (148.92pt, -629.2pt) -- cycle
;
\end{scope}
\end{tikzpicture}
\begin{picture}(-5,0)(2.5,0)
\put(148.9201,-626.3199){\fontsize{11.891}{1}\usefont{T1}{ptm}{m}{n}\selectfont\color{color_29791}3}
\put(148.9201,-643.1199){\fontsize{11.891}{1}\usefont{T1}{ptm}{m}{n}\selectfont\color{color_29791}2}
\end{picture}
\begin{tikzpicture}[overlay]
\path(0pt,0pt);
\begin{scope}
\clip
(146.28pt, -615.28pt) -- (158.28pt, -615.28pt)
 -- (158.28pt, -615.28pt)
 -- (158.28pt, -646pt)
 -- (158.28pt, -646pt)
 -- (146.28pt, -646pt) -- cycle
;
\draw[color_29791,line width=0.24pt,line cap=round,line join=round]
(146.28pt, -615.28pt) -- (146.28pt, -615.28pt)
;
\end{scope}
\end{tikzpicture}
\begin{picture}(-5,0)(2.5,0)
\put(157.8,-635.92){\fontsize{12}{1}\usefont{T1}{ptm}{m}{n}\selectfont\color{color_63426} to solve equation (4) and }
\end{picture}
\begin{tikzpicture}[overlay]
\path(0pt,0pt);
\begin{scope}
\clip
(267.96pt, -615.28pt) -- (286.68pt, -615.28pt)
 -- (286.68pt, -615.28pt)
 -- (286.68pt, -646pt)
 -- (286.68pt, -646pt)
 -- (267.96pt, -646pt) -- cycle
;
\draw[color_29791,line width=0.57pt,line cap=round,line join=round]
(269.8795pt, -630.8798pt) -- (283.8pt, -630.8798pt)
;
\end{scope}
\begin{scope}
\clip
(277.32pt, -615.52pt) -- (286.68pt, -615.52pt)
 -- (286.68pt, -615.52pt)
 -- (286.68pt, -629.2pt)
 -- (286.68pt, -629.2pt)
 -- (277.32pt, -629.2pt) -- cycle
;
\end{scope}
\end{tikzpicture}
\begin{picture}(-5,0)(2.5,0)
\put(277.3202,-626.3199){\fontsize{11.891}{1}\usefont{T1}{ptm}{m}{n}\selectfont\color{color_29791}7}
\put(273.9602,-643.1199){\fontsize{11.891}{1}\usefont{T1}{ptm}{m}{n}\selectfont\color{color_29791}5}
\put(270.8402,-626.3199){\fontsize{11.891}{1}\usefont{T1}{ptm}{m}{n}\selectfont\color{color_29791}−}
\end{picture}
\begin{tikzpicture}[overlay]
\path(0pt,0pt);
\begin{scope}
\clip
(267.96pt, -615.28pt) -- (286.68pt, -615.28pt)
 -- (286.68pt, -615.28pt)
 -- (286.68pt, -646pt)
 -- (286.68pt, -646pt)
 -- (267.96pt, -646pt) -- cycle
;
\draw[color_29791,line width=0.24pt,line cap=round,line join=round]
(267.96pt, -615.28pt) -- (267.96pt, -615.28pt)
;
\end{scope}
\end{tikzpicture}
\begin{picture}(-5,0)(2.5,0)
\put(286.2,-635.92){\fontsize{12}{1}\usefont{T1}{ptm}{m}{n}\selectfont\color{color_63426} to solve}
\put(51.00002,-662.32){\fontsize{12}{1}\usefont{T1}{ptm}{m}{n}\selectfont\color{color_63426}equation (5). This leads us to the collection of }
\put(239.64,-662.32){\fontsize{12}{1}\usefont{T1}{ptm}{m}{n}\selectfont\color{color_63426}rational numbers}
\put(318.84,-662.32){\fontsize{12}{1}\usefont{T1}{ptm}{m}{n}\selectfont\color{color_63426}.}
\put(69.00003,-684.4){\fontsize{12}{1}\usefont{T1}{ptm}{m}{n}\selectfont\color{color_63426}W}
\put(79.56003,-684.4){\fontsize{12}{1}\usefont{T1}{ptm}{m}{n}\selectfont\color{color_63426}e have already seen basic operations on rational}
\put(51.00003,-700.72){\fontsize{12}{1}\usefont{T1}{ptm}{m}{n}\selectfont\color{color_63426}numbers. }
\put(93.96003,-700.72){\fontsize{12}{1}\usefont{T1}{ptm}{m}{n}\selectfont\color{color_63426}W}
\put(104.04,-700.72){\fontsize{12}{1}\usefont{T1}{ptm}{m}{n}\selectfont\color{color_63426}e now try to explore some properties of operations}
\put(51.00003,-716.8){\fontsize{12}{1}\usefont{T1}{ptm}{m}{n}\selectfont\color{color_63426}on the dif}
\put(93.48003,-716.8){\fontsize{12}{1}\usefont{T1}{ptm}{m}{n}\selectfont\color{color_63426}ferent types of numbers seen so far}
\put(249.72,-716.8){\fontsize{12}{1}\usefont{T1}{ptm}{m}{n}\selectfont\color{color_63426}.}
\end{picture}
\begin{tikzpicture}[overlay]
\path(0pt,0pt);
\filldraw[color_283006][even odd rule]
(360.5784pt, -66.92078pt) -- (523.319pt, -66.92078pt)
 -- (523.319pt, -66.92078pt)
 -- (523.319pt, -47.58112pt)
 -- (523.319pt, -47.58112pt)
 -- (360.5784pt, -47.58112pt) -- cycle
;
\draw[color_283006,line width=0.96pt,miter limit=1000]
(360.5784pt, -66.92078pt) -- (523.319pt, -66.92078pt)
 -- (523.319pt, -66.92078pt)
 -- (523.319pt, -47.58112pt)
 -- (523.319pt, -47.58112pt)
 -- (360.5784pt, -47.58112pt) -- cycle
;
\filldraw[color_248286][even odd rule]
(51pt, -186.2008pt) -- (447.9984pt, -186.2008pt)
 -- (447.9984pt, -186.2008pt)
 -- (447.9984pt, -98.45081pt)
 -- (447.9984pt, -98.45081pt)
 -- (51pt, -98.45081pt) -- cycle
;
\end{tikzpicture}
\begin{picture}(-5,0)(2.5,0)
\put(103.8,-148.72){\fontsize{32}{1}\usefont{T1}{ptm}{m}{n}\selectfont\color{color_63426}Rational Numbers}
\end{picture}
\begin{tikzpicture}[overlay]
\path(0pt,0pt);
\draw[color_35424,line width=1.92pt,miter limit=1000]
(73.65pt, -93.48883pt) -- (63.69pt, -93.48883pt)
 -- (63.69pt, -93.48883pt)
 -- (63.69pt, -192.6616pt)
 -- (63.69pt, -192.6616pt)
 -- (73.65pt, -192.6616pt)
;
\draw[color_63426,line width=0.96pt]
(69pt, -99.28003pt) -- (69pt, -186.88pt)
;
\draw[color_35424,line width=1.92pt,miter limit=1000]
(425.2896pt, -92.74719pt) -- (435.2472pt, -92.74719pt)
 -- (435.2472pt, -92.74719pt)
 -- (435.2472pt, -191.9152pt)
 -- (435.2472pt, -191.9152pt)
 -- (425.2896pt, -191.9152pt)
;
\draw[color_63426,line width=0.96pt]
(429.72pt, -98.56pt) -- (429.72pt, -186.16pt)
;
\end{tikzpicture}
\begin{picture}(-5,0)(2.5,0)
\put(455.4,-110.08){\fontsize{16}{1}\usefont{T1}{ptm}{m}{n}\selectfont\color{color_63426}CHAPTER}
\put(458.52,-189.28){\fontsize{100}{1}\usefont{T1}{ptm}{m}{n}\selectfont\color{color_35424}1}
\put(442.4496,-281.2){\includegraphics[width=84.94991pt,height=84.94991pt]{latexImage_74592d3fc4a22e8f5b059c186ef44901.png}}
\put(272.72,-767.116){\fontsize{8}{1}\usefont{T1}{ptm}{m}{n}\selectfont\color{color_29791}2019-20}
\put(2.360001,-795.9999){\includegraphics[width=570.0885pt,height=806.3993pt]{latexImage_87010b2d2d6dcd6c6342a1e9d8ab4a73.png}}
\put(62.35721,-641.7543){\includegraphics[width=435.3821pt,height=435.3821pt]{latexImage_25f6cc883460785699e26952c3e0b750.png}}
\end{picture}
\end{document}